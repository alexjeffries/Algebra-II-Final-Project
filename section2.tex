\section{Properties of an OHFD}

\subsection{Definitions}

\begin{frame}
  \frametitle{Mutually Nondegenerate Irreducible Factorizations}
  \begin{definition}
    Let $R$ be an integral domain. We say that a pair of irreducible factorizations
      $$\pi_1\pi_2\cdots\pi_n = \xi_1\xi_2\cdots\xi_m$$
    is \alert{mutually nondegenerate} if the irreducibles $\pi_i \in R$ and $\xi_j \in R$ are pairwise non-associate.
  \end{definition}
  \pause{}
  \begin{example}
    Let $\pi_1 \pi_2 = \xi_1 \xi_2$ be a pair of irreducible factorizations in a finite integral domain such that $\pi_1$ and $\xi_1$ are associate.
    Then $\pi_1 = \xi_1 u$ where $u$ is unit in the domain.
    We can write $\xi_1 u \pi_2 = \xi_1 \xi_2$ then $u \pi_2 = \xi_2$.
    In this sense, the factorization is degenerate.
  \end{example}
\end{frame}

\begin{frame}
  \frametitle{Long/Short}
  \begin{definition}
    Let $R$ be an atomic domain and $\pi_1 \in R$ an irreducible element.
    We say that $\pi_1$ is \alert{long} (resp. \alert{short}) if there is a pair of mutually nondegenerate irreducible factorizations
      $$\pi_1\pi_2\cdots\pi_n = \xi_1\xi_2\cdots \xi_m$$
    with $\pi_i, \xi_j \in R$ irreducible and $n > m$ (resp. $n<m$)
  \end{definition}
\end{frame}

\subsection{Propositions}

\begin{frame}
  \frametitle{Lemma 1}
  \begin{lemma}
    Suppose that $R$ is an OHFD and let
      $$\pi_1\cdots\pi_k=\xi_1\cdots\xi_m$$
      and
      $$\alpha_1\cdots\alpha_r=\beta_1\cdots\beta_n$$
    be two pairs of mutually nondegenerate irreducible factorizations with $k>m$ and $n>r$.
    Then each $\pi_i$ is associate to some $\beta_j$ and conversely (likewise for $\xi_i$ and $\alpha_j$). 
  \end{lemma}
\end{frame}

\begin{frame}
  \frametitle{Proof of Lemma 1}
  \begin{proof}
    \begin{itemize}
      \item <1->The proof uses the fact that equal length factorizations are unique.
      \item <2->Let $a = n-r$ and $b = k - m$. In the factorizations:
	  $$\pi_1^a\cdots\pi_k^a\alpha_1^b\cdots\alpha_r^b=\xi_1^a\cdots\xi_m^a\beta_1^b\cdots\beta_n^b$$
	the lengths are the same since $ak+br=am+bn$. 
      \item <3->
	Since $R$ is a OHFD, each $\pi_i$ must appear on the right somewhere.
      \item<4->
	Since the $\pi_i$'s and $\xi_j$'s are non-associate, this means that each $\pi_i$ is an associate of one of the $\beta_j$'s. 
      \item<5->The same argument shows that the $\alpha_i$'s and $\xi_j$'s are associates as well.
    \end{itemize}
  \end{proof}
\end{frame}

\begin{frame}
  \frametitle{Lemma 2}
  \begin{lemma}
    Let $R$ be an OHFD that is not a HFD and $\pi \in R$ a nonprime irreducible.
    Then $\pi$ cannot be both long and short.
  \end{lemma}
\end{frame}

\begin{frame}
  \frametitle{Proof of Lemma 2}
  \begin{proof}
    \begin{itemize}
      \item <1->Assume that $\pi_1$ is both long and short.
      \item <2->Then, we have two mutually nondegenerate irreducible factorizations of the following form:
	  $$\pi_1\cdots\pi_k=\xi_1\cdots\xi_m \quad k > m$$
	and
	  $$\pi_1\alpha_2\cdots\alpha_r=\beta_1\cdots\beta_n \quad r < n$$
      \item <3->By the previous Lemma, this implies that $\pi_1$ is an associate of one of the $\xi_j$'s.
	This contradicts the mutual nondegeneracy of the first pair of factorizations.
    \end{itemize}
  \end{proof}
\end{frame}

\begin{frame}
  \frametitle{Finitely Many Long/Short}
  \begin{fact}
    Let $R$ be an OHFD, then $R$ has only finitely many long (resp. short) irreducibles.
  \end{fact}
\end{frame}

\section{The Main Result}

\subsection{Some Preliminaries }

\begin{frame}
  \frametitle{Master Factorization }
  \begin{definition}
    Let $R$ be an OHFD, and let $\{\pi_{1},\pi_{2},\ldots,\pi_{k}\}$ and $\{\xi_{1},\xi_{2},\ldots,\xi_{m}\}$ respectively denote the sets of long and short irreducibles in $R$ (up to associates).
    Among all pairs of factorizations
      $$\pi_{1}^{a_{1}}\pi_{2}^{a_{2}}\cdots\pi_{k}^{a_{k}}=  \xi_{1}^{b_{1}}\xi_{2}^{b_{2}}\cdots\xi_{m}^{b_{m}}$$
    we select one with $a_{1}$ minimal and call this pair of factorizations the \alert{master factorization (MF)}.
  \end{definition}
\end{frame}

\begin{frame}
  \frametitle{An Interesting Result }
  \begin{fact}
    Let $R$ be a an OHFD and let

      $$\pi_{1}^{a_{1}}\pi_{2}^{a_{2}}\cdots\pi_{k}^{a_{k}}=  \xi_{1}^{b_{1}}\xi_{2}^{b_{2}}\cdots\xi_{m}^{b_{m}}$$

    be the MF.
    Then any pair of mutually irreducible factorizations of this MF are powers.
    That is, any two mutually nondegenerate irreducible factorizations of different lengths are of the form (up to associates):

      $$\pi_{1}^{ta_{1}}\pi_{2}^{ta_{2}}\cdots\pi_{k}^{ta_{k}}=  \xi_{1}^{tb_{1}}\xi_{2}^{tb_{2}}\cdots\xi_{m}^{tb_{m}}$$

    for some $t\geq1$.
  \end{fact}
\end{frame}

\subsection{The Main Result}
\begin{frame}
  \frametitle{The Result}
  \begin{theorem}
    If $R$ is an OHFD, then it is an HFD.
  \end{theorem}
\end{frame}

\begin{frame}
  \frametitle{An overview of the Proof}
  \begin{itemize}
    \item To prove this, we assume that $R$ is an OHFD, but not an HFD, then show that the MF must not exist, which is a contradiction.
    \item This is done in three cases, but we only discuss one case.
  \end{itemize}
\end{frame}

\begin{frame}
  \frametitle{The Proof}
  \begin{proof}
    First, assume that $R$ is and OHFD that is not an HFD. Let the MF of $R$ be 

      $$\pi_{1}^{a_{1}}\pi_{2}^{a_{2}}\cdots\pi_{k}^{a_{k}}=  \xi_{1}^{b_{1}}\xi_{2}^{b_{2}}\cdots\xi_{m}^{b_{m}}$$

    with $\sum_{i=1}^{k}a_{i}>\sum_{i=1}^{m}b_{i}$
    \noqedsymbol
  \end{proof}
\end{frame}

\begin{frame}
  \frametitle{The Proof}
  \begin{proof}
    \begin{itemize}
      \item<1-> Of the three cases, we consider the case where $k,m \geq 2$
      \item<2-> Consider the element $(\pi_{1}^{a_{1}}-\xi_{1}^{b_{1}})(\pi_{1}^{a_{1}}\pi_{2}^{a_{2}}\cdots\pi_{k}^{a_{k}}-\xi_{2}^{b_{2}}\xi_{3}^{b_{3}}\cdots\xi_{m}^{b_{m}})$
      \item<3-> Noting that $\xi_{1}^{b_{1}}\xi_{2}^{b_{2}}\cdots\xi_{m}^{b_{m}}=\pi_{1}^{a_{1}}\pi_{2}^{a_{2}}\cdots\pi_{k}^{a_{k}}$, we see that $\pi_{1}$divides this product. 
      \item<4-> Not as easy to see, $\pi_{1}$divides neither $\xi_{1}^{b_{1}}$ nor $\xi_{2}^{b_{2}}\xi_{3}^{b_{3}}\cdots\xi_{m}^{b_{m}}$
    \end{itemize}
    \noqedsymbol
  \end{proof}
\end{frame}

\begin{frame}
  \frametitle{The Proof}
  \begin{proof}
    \begin{itemize}
      \item<1-> If $\pi_{1}$does divide $\xi_{2}^{b_{2}}\xi_{3}^{b_{3}}\cdots\xi_{m}^{b_{m}}$ then there is some $c\in R$ such that $c\pi_{1}=\xi_{2}^{b_{2}}\xi_{3}^{b_{3}}\cdots\xi_{m}^{b_{m}}$
      \item<2-> Since $\pi_1$ is not an associate of any of the $\xi_j$'s, after factoring $c$, we see that the factorizations are of unequal lengths.
      \item<3-> By the fact we mentioned, we have $\xi_{2}^{b_{2}}\xi_{3}^{b_{3}}\cdots\xi_{m}^{b_{m}}=\xi_{1}^{tb_{1}}\xi_{2}^{tb_{2}}\cdots\xi_{m}^{tb_{m}}$ for some $t$, which implies that $\xi_{1}$ divides $\xi_{2}^{b_{2}}\xi_{3}^{b_{3}}\cdots\xi_{m}^{b_{m}}$.
    \end{itemize}
    \noqedsymbol
  \end{proof}
\end{frame}

\begin{frame}
  \frametitle{The Proof}
  \begin{proof}
    \begin{itemize}
      \item<1-> So we have some $d\in R$ such that $d\xi_{1}=\xi_{2}^{b_{2}}\xi_{3}^{b_{3}}\cdots\xi_{m}^{b_{m}}$
      \item<2-> Given that $\xi_{1}$ is short, the above factorizations must be either mutually degenerate or of equal lengths.
      \item<3-> Either way, $\xi_{1}$ is an associate of one of the $\xi_{i}$'s on the RHS. This is a contradiction so $\pi_{1}$does not divide $\xi_{2}^{b_{2}}\xi_{3}^{b_{3}}\cdots\xi_{m}^{b_{m}}$
    \end{itemize}
    \noqedsymbol
  \end{proof}
\end{frame}

\begin{frame}
  \frametitle{The Proof}
  \begin{proof}
    \begin{itemize}
      \item<1-> So, $(\pi_{1}^{a_{1}}-\xi_{1}^{b_{1}})(\pi_{1}^{a_{1}}\pi_{2}^{a_{2}}\cdots\pi_{k}^{a_{k}}-\xi_{2}^{b_{2}}\xi_{3}^{b_{3}}\cdots\xi_{m}^{b_{m}})=k\pi_{1}$
      \item<2-> We can factor $(\pi_{1}^{a_{1}}-\xi_{1}^{b_{1}})$ into $\alpha_{1}\ldots\alpha_{s}$ and $(\pi_{1}^{a_{1}}\pi_{2}^{a_{2}}\cdots\pi_{k}^{a_{k}}-\xi_{2}^{b_{2}}\xi_{3}^{b_{3}}\cdots\xi_{m}^{b_{m}})$ into $\beta_{1}\ldots\beta_{t}$where the $\alpha_{1}$'s and $\beta_{i}$'s are irreducible. 
      \item<3-> Thus, $\alpha_{1}\cdots\alpha_{s}$$\beta_{1}\cdots\beta_{t}=k\pi_{1}$ 
    \end{itemize}
    \noqedsymbol
  \end{proof}
\end{frame}

\begin{frame}
  \frametitle{The Proof}
  \begin{proof}
    \begin{itemize}
      \item<1-> From the previous argument, $\pi_{1}$ is not associated to any of the $\alpha_{1},\ldots,\alpha_{s}$ or $\beta_{1},\ldots,\beta_{t}$. 
      \item<2-> So, these are factorizations of different lengths, and in the factorization:
	  $$(\pi_{1}^{a_{1}}-\xi_{1}^{b_{1}})(\pi_{1}^{a_{1}}\pi_{2}^{a_{2}}\cdots\pi_{k}^{a_{k}}-\xi_{2}^{b_{2}}\xi_{3}^{b_{3}}\cdots\xi_{m}^{b_{m}}) =k\pi_{1}$$
	The RHS is the long side. 
      \item<3-> Again, we use the fact we stated and observe that $\xi_{1}$divides $\alpha_{1}\cdots\alpha_{s}$$\beta_{1}\cdots\beta_{t}$, and in particular, $\xi_{1}$divides $(\pi_{1}^{a_{1}}-\xi_{1}^{b_{1}})$ or $(\pi_{1}^{a_{1}}\pi_{2}^{a_{2}}\cdots\pi_{k}^{a_{k}}-\xi_{2}^{b_{2}}\xi_{3}^{b_{3}}\cdots\xi_{m}^{b_{m}}).$
      \item<4-> This means that $\xi_{1}$divides $\pi_{1}^{a_{1}}$ or $\xi_{2}^{b_{2}}\xi_{3}^{b_{3}}\cdots\xi_{m}^{b_{m}}$.
    \end{itemize}
    \noqedsymbol
  \end{proof}
\end{frame}

\begin{frame}
  \frametitle{The Proof}
  \begin{proof}
    \begin{itemize}
      \item<1-> If $\xi_{1}$ divides $\pi_{1}^{a_{1}}$ then $\pi_{1}^{a_{1}}=c\xi_{1}$.
	The factorizations are of different lengths.
	As we have done before, this leads to a contradiction since $\pi_{1}$ is not associated to the other $\pi_{i}$'s. 
      \item<2-> Otherwise, $c\xi_{1}=\xi_{2}^{b_{2}}\xi_{3}^{b_{3}}\cdots\xi_{m}^{b_{m}}$
      \item<3-> This means that $\xi_{1}$is associated with one of the $\xi_{i}$'s which is a contradiction.
    \end{itemize}
    \noqedsymbol
  \end{proof}
\end{frame}

\begin{frame}
  \frametitle{The Proof}
  \begin{proof}
    \begin{itemize}
      \item<1-> Therefore, we have no MFs of this form.
      \item<2-> The proof for the other two cases are similar. 
    \end{itemize}
  \end{proof}
\end{frame}

\subsection{Consequences}

\begin{frame}
  \frametitle{A Shocking Result}
  \begin{theorem}
    An integral domain is an OHFD if and only if it as a UFD.
  \end{theorem}
  \pause{}
  \begin{proof}
    \begin{itemize}
      \item<1-> The $(\Leftarrow)$ direction is trivial.
      \item<2-> By the previous theorem, an OHFD is a HFD. 
      \item<3-> These two together imply that it is a UFD.
    \end{itemize}
  \end{proof}
\end{frame}

\begin{frame}
  \frametitle{A New Definition}
  \begin{fact}
    We can thus modify the definition of a UFD to only include the OHFD axiom.
  \end{fact}
  \pause{}
  \begin{definition}
    An integral domain, $R$, is a \alert{unique factorization domain (UFD)} if every nonzero nonunit element in $R$ can be factored into irreducible elements, and if we have
    $$ \alpha_1 \alpha_2 \cdots \alpha_n = \beta_1 \beta_2 \cdots \beta_m $$
    with each $\alpha_i, \beta_j$ irreducible in $R$, then
    \begin{enumerate}[(a)]
      \item There is a $\sigma \in S_n$ such that $\alpha_i = u_i \beta_{\sigma(i)}$ for all $1 \leq i \leq n$ where each $u_i$ is a unit of $R$.
    \end{enumerate}
  \end{definition}
\end{frame}
