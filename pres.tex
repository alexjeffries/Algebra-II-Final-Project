%% LyX 1.6.5 created this file.  For more info, see http://www.lyx.org/.
%% Do not edit unless you really know what you are doing.
\documentclass[english]{beamer}
\usepackage{mathptmx}
\usepackage[T1]{fontenc}
\usepackage[latin9]{inputenc}
\usepackage{amsmath}
\usepackage{amssymb}
%%TEST1
\makeatletter

%%%%%%%%%%%%%%%%%%%%%%%%%%%%%% LyX specific LaTeX commands.
\DeclareRobustCommand*{\lyxarrow}{%
\@ifstar
{\leavevmode\,$\triangleleft$\,\allowbreak}
{\leavevmode\,$\triangleright$\,\allowbreak}}

%%%%%%%%%%%%%%%%%%%%%%%%%%%%%% Textclass specific LaTeX commands.
 % this default might be overridden by plain title styl
 \newcommand\makebeamertitle{\frame{\maketitle}}%
 \AtBeginDocument{
   \let\origtableofcontents=\tableofcontents
   \def\tableofcontents{\@ifnextchar[{\origtableofcontents}{\gobbletableofcontents}}
   \def\gobbletableofcontents#1{\origtableofcontents}
 }
 \makeatletter
 \long\def\lyxframe#1{\@lyxframe#1\@lyxframestop}%
 \def\@lyxframe{\@ifnextchar<{\@@lyxframe}{\@@lyxframe<*>}}%
 \def\@@lyxframe<#1>{\@ifnextchar[{\@@@lyxframe<#1>}{\@@@lyxframe<#1>[]}}
 \def\@@@lyxframe<#1>[{\@ifnextchar<{\@@@@@lyxframe<#1>[}{\@@@@lyxframe<#1>[<*>][}}
 \def\@@@@@lyxframe<#1>[#2]{\@ifnextchar[{\@@@@lyxframe<#1>[#2]}{\@@@@lyxframe<#1>[#2][]}}
 \long\def\@@@@lyxframe<#1>[#2][#3]#4\@lyxframestop#5\lyxframeend{%
   \frame<#1>[#2][#3]{\frametitle{#4}#5}}
 \makeatother
 \def\lyxframeend{} % In case there is a superfluous frame end

%%%%%%%%%%%%%%%%%%%%%%%%%%%%%% User specified LaTeX commands.
\usetheme{Warsaw}
% or ...

\setbeamercovered{transparent}
% or whatever (possibly just delete it)

\makeatother

\usepackage{babel}

\begin{document}





\title[On UFD's]{On Unique Factorization Domains}


\author{Jim Coykendall\inst{1} \and William W. Smith\inst{2}}


\institute{\inst{1}Department Mathematics\\
North Dakota State University\and \inst{2}Department of Mathematics\\
The University of North Carolina}


\date{Algebra II Final Project, 2011}

\makebeamertitle


%\pgfdeclareimage[height=0.5cm]{institution-logo}{institution-logo-filename}

%\logo{\pgfuseimage{institution-logo}}



\AtBeginSubsection[]{

  \frame<beamer>{ 

    \frametitle{Outline}   

    \tableofcontents[currentsection,currentsubsection] 

  }

}



%\beamerdefaultoverlayspecification{<+->}


\lyxframeend{}\lyxframe{Outline}

\tableofcontents{}




\lyxframeend{}\section{Motivation}


\lyxframeend{}\subsection[BasProblem]{The Basic Problem That We Studied}


\lyxframeend{}\lyxframe{Make Titles Informative. Use Uppercase Letters.}


\framesubtitle{Frame subtitles are optional. Use upper- or lowercase letters.}
\begin{itemize}
\item Use Itemize a lot.


\pause{}

\item Use very short sentences or short phrases.


\pause{}

\item These overlays are created using the Pause style.
\end{itemize}

\lyxframeend{}\lyxframe{Make Titles Informative. }
\begin{itemize}
\item <1->You can also use overlay specifications to create overlays.
\item <3->This allows you to present things in any order.
\item <2->This is shown second.
\end{itemize}

\lyxframeend{}\lyxframe{Make Titles Informative.}
\begin{block}
<1->{}
\begin{itemize}
\item Untitled block.
\item Shown on all slides.
\end{itemize}
\end{block}
\begin{exampleblock}
<2->{Some Example Block Title}
\begin{itemize}
\item $e^{i\pi}=-1$.
\item $e^{i\pi/2}=i$.
\end{itemize}
\end{exampleblock}

\lyxframeend{}\subsection{Previous Work}


\lyxframeend{}\lyxframe{Make Titles Informative. }
\begin{example}%{}
<1->On first slide. 
\end{example}%{}

\begin{example}%{}
<2->On second slide.


\end{example}%{}

\lyxframeend{}\section{Properties of an OHFD}


\lyxframeend{}\subsection{Definitions}


\lyxframeend{}\lyxframe{Mutually Nondegenerate Irreducible Factorizations}
\begin{definition}
Let $R$ be an integral domain. We say that a pair of irreducible factorizations
\begin{align*}
\pi_1\pi_2\cdots\pi_n = \xi_1\xi_2\cdots\xi_m
\end{align*}
is mutually nondegenerate if the irreducibles $\pi_1 \in R$ and $\xi \in R$ are pairwise non-associate.
\end{definition}

\lyxframeend{}\lyxframe{Long/Short}

\begin{definition}
Let $R$ be an atomic domain and $\pi_1 \in R$ an irreducible element. We say that $\pi_1$ is "long" (resp. "short") if there is a pair of mutually nondegenerate irredicible factorizations
\begin{align*}
\pi_1\pi_2\cdots\pi_n = \xi_1\xi_2\cdots \xi_m
\end{align*}
with $\pi_i, \xi_j \in R$ irreducible and $n > m$ (resp. $n<m$)
\end{definition}


\lyxframeend{}\subsection{Propositions}
\lyxframe{Lemma 1}
\begin{lemma}
Suppose that $R$ is an OHFD and let
\begin{align*}
\pi_1\cdots\pi_k=\xi_1\cdots\xi_m
\end{align*}
and
\begin{align*}
\alpha_1\cdots\alpha_r=\beta_1\cdots\beta_n
\end{align*}
be two pairs of mutually nondegenerate irreducible factorizations with $k>m$ and $n>r$. Then each $\pi_i$ is associate to some $\beta_j$ and each $\xi_1$ is associate to some $\alpha_j$. 
\end{lemma}
\lyxframeend{}\lyxframe{Proof}
\begin{itemize}
\item <1->The proof uses the fact that equal length factorizations are unique.
\item <2->Let $a = n-r$ and $b = k - m$. In the factorizations:
\begin{align*}
\pi_1^a\cdots\pi_k^a\alpha_1^b\cdots\alpha_r^b=\xi_1^a\cdots\xi_m^a\beta_1^b\cdots\beta_n^b
\end{align*}
the lengths are the same since $ak+br=am+bn$. 
\item <3->Therefore, each $\pi_i$ must appear on the right somewhere. Since the $\pi_i$'s and $\xi_j$'s are non-associate, this means that each $\pi_i$ is an associate of one of the $\beta_j$'s. 
\item<4->The same argument shows that the $\alpha_i$'s and $\xi_j$'s are associates. 
\end{itemize}

\lyxframeend{}\lyxframe{Lemma 2}
\begin{lemma}
Let $R$ be an OHFD that is not a HFD and $\pi \in R$ a nonprime irreducible. Then $\pi$ cannot be both long and short.
\end{lemma}


\lyxframeend{}\lyxframe{Proof}
\begin{itemize}
\item <1->Assume that $\pi_1$ is both long and short.
\item <2->Then, we have two mutually nondegenerate irreducible factorizations of the following form:
\begin{align*}
\pi_1\cdots\pi_k=\xi_1\cdots\xi_m
\end{align*}
and
\begin{align*}
\pi_1\alpha_2\cdots\alpha_r=\beta_1\cdots\beta_n
\end{align*}
\item <3->By the previous Lemma, this implies that $\pi_1$ is an associate of one of the $\xi_j$'s. This contradicts the mutual nondegeneracy of the first pair of factorizations.
\end{itemize}


\lyxframeend{}\section{The Main Result}


\lyxframeend{}\subsection{Some Preliminaries }


\lyxframeend{}\lyxframe{Master Factorization }
\begin{definition}%{}
Let $R$ be an OHFD, and let $\{\pi_{1},\pi_{2},\ldots,\pi_{k}\}$and
$\{\xi_{1},\xi_{2},\ldots,\xi_{m}\}$ respictevly denote the sets
of long and short irreducibles in $R$ (up to associates). Among all
pairs of factorizations

\begin{align*}
\pi_{1}^{a_{1}}\pi_{2}^{a_{2}}\cdots\pi_{k}^{a_{k}}= & \xi_{1}^{b_{1}}\xi_{2}^{b_{2}}\cdots\xi_{m}^{b_{m}}\end{align*}


we select one with $a_{1}$ minimal and call this pair of factorizations
the master factorization.


\end{definition}%{}

\lyxframeend{}\lyxframe{An Interesting Result }
\begin{fact}%{}
Let $R$ be a an OHFD and let

\begin{align*}
\pi_{1}^{a_{1}}\pi_{2}^{a_{2}}\cdots\pi_{k}^{a_{k}}= & \xi_{1}^{b_{1}}\xi_{2}^{b_{2}}\cdots\xi_{m}^{b_{m}}\end{align*}


be the MF. Then any pair of mutually irreducible factorizations of
this MF. That is, any two mutually nondegenerate irreducible factroizations
of different lenghts are of the form (up to associates):

\begin{align*}
\pi_{1}^{ta_{1}}\pi_{2}^{ta_{2}}\cdots\pi_{k}^{ta_{k}}= & \xi_{1}^{tb_{1}}\xi_{2}^{tb_{2}}\cdots\xi_{m}^{tb_{m}}\end{align*}


for some $t\geq1$.
\end{fact}%{}

\lyxframeend{}\subsection{The Main Result}


\lyxframeend{}\lyxframe{The Proof}
\begin{itemize}
\item To prove this, we assume that $R$ is an OHFD, but not an HFD, then
show that the MF must not exist, which is a contradiction.
\item This is done in three cases, but we only discuss one case.
\end{itemize}

\lyxframeend{}\lyxframe{The Proof}

First, assume that $R$ is and OHFD that is not an HFD. Let the MF
of $R$ be 

\begin{align*}
\pi_{1}^{a_{1}}\pi_{2}^{a_{2}}\cdots\pi_{k}^{a_{k}}= & \xi_{1}^{b_{1}}\xi_{2}^{b_{2}}\cdots\xi_{m}^{b_{m}}\end{align*}


with $\sum_{i=1}^{k}a_{i}>\sum_{i=1}^{m}b_{i}$


\lyxframeend{}\lyxframe{The Proof}
\begin{itemize}
\item Of the three cases, we consider the case where $k,m>2$
\end{itemize}

\pause{}
\begin{itemize}
\item Consider the element $(\pi_{1}^{a_{1}}-\xi_{1}^{b_{1}})(\pi_{1}^{a_{1}}\pi_{2}^{a_{2}}\cdots\pi_{k}^{a_{k}}-\xi_{2}^{b_{2}}\xi_{3}^{b_{3}}\cdots\xi_{m}^{b_{m}})$
\end{itemize}

\pause{}
\begin{itemize}
\item Noting that $\xi_{1}^{b_{1}}\xi_{2}^{b_{2}}\cdots\xi_{m}^{b_{m}}=\pi_{1}^{a_{1}}\pi_{2}^{a_{2}}\cdots\pi_{k}^{a_{k}}$,
we see that $\pi_{1}$divides this product. 
\end{itemize}

\pause{}
\begin{itemize}
\item Not as easy to see, $\pi_{1}$divides neither $\xi_{1}^{b_{1}}$ nor
$\xi_{2}^{b_{2}}\xi_{3}^{b_{3}}\cdots\xi_{m}^{b_{m}}$
\end{itemize}

\lyxframeend{}\lyxframe{The Proof}
\begin{itemize}
\item If $\pi_{1}$does divide $\xi_{2}^{b_{2}}\xi_{3}^{b_{3}}\cdots\xi_{m}^{b_{m}}$
then there is some $c\in R$ such that $c\pi_{1}=\xi_{2}^{b_{2}}\xi_{3}^{b_{3}}\cdots\xi_{m}^{b_{m}}$
\end{itemize}

\pause{}
\begin{itemize}
\item If we factor $c$ we see that the factorizations are of unequal lengths;
$\pi_{1}$ is long so the LHS of this is the long side.
\end{itemize}

\pause{}
\begin{itemize}
\item By the fact we mentioned, we have $\xi_{2}^{b_{2}}\xi_{3}^{b_{3}}\cdots\xi_{m}^{b_{m}}=\xi_{1}^{tb_{1}}\xi_{2}^{tb_{2}}\cdots\xi_{m}^{tb_{m}}$
for some $t$, which implies that $\xi_{1}$ divides $\xi_{2}^{b_{2}}\xi_{3}^{b_{3}}\cdots\xi_{m}^{b_{m}}$.
\end{itemize}

\lyxframeend{}\lyxframe{The Proof}
\begin{itemize}
\item So we have some $d\in R$ such that $d\xi_{1}=\xi_{2}^{b_{2}}\xi_{3}^{b_{3}}\cdots\xi_{m}^{b_{m}}$
\end{itemize}

\pause{}
\begin{itemize}
\item Given that $\xi_{1}$ is short, the above factorizations must be either
mutually degenerate or of equal lengths.
\end{itemize}

\pause{}
\begin{itemize}
\item Either, way, $\xi_{1}$ is an associate of one of the $\xi_{i}'s$
on the RHS. This is a contradiction so $\pi_{1}$does not divide $\xi_{2}^{b_{2}}\xi_{3}^{b_{3}}\cdots\xi_{m}^{b_{m}}$
\end{itemize}

\lyxframeend{}\lyxframe{The Proof}
\begin{itemize}
\item So, $(\pi_{1}^{a_{1}}-\xi_{1}^{b_{1}})(\pi_{1}^{a_{1}}\pi_{2}^{a_{2}}\cdots\pi_{k}^{a_{k}}-\xi_{2}^{b_{2}}\xi_{3}^{b_{3}}\cdots\xi_{m}^{b_{m}})=k\pi_{1}$
\end{itemize}

\pause{}
\begin{itemize}
\item We can factor $(\pi_{1}^{a_{1}}-\xi_{1}^{b_{1}})$ into $\alpha_{1},\ldots,\alpha_{s}$
and $(\pi_{1}^{a_{1}}\pi_{2}^{a_{2}}\cdots\pi_{k}^{a_{k}}-\xi_{2}^{b_{2}}\xi_{3}^{b_{3}}\cdots\xi_{m}^{b_{m}})$
into $\beta_{1},\ldots,\beta_{t}$where the $\alpha_{1}'s$ and $\beta_{i}'s$
are irreducible. 
\end{itemize}

\pause{}
\begin{itemize}
\item Thus, $\alpha_{1}\cdots\alpha_{s}$$\beta_{1}\cdots\beta_{t}=k\pi_{1}$ 
\end{itemize}

\lyxframeend{}\lyxframe{The Proof}
\begin{itemize}
\item Since $\pi_{1}$is not associated to any other $\pi_{i}'s$ from the
previous argument, $\pi_{1}$ is not associated to any of the $\alpha_{1},\ldots,\alpha_{s}$or
$\beta_{1},\ldots,\beta_{t}$. 
\end{itemize}

\pause{}
\begin{itemize}
\item So, these are factorizations of different lengths, and in the factorization:
\begin{align*}
(\pi_{1}^{a_{1}}-\xi_{1}^{b_{1}})(\pi_{1}^{a_{1}}\pi_{2}^{a_{2}}\cdots\pi_{k}^{a_{k}}-\xi_{2}^{b_{2}}\xi_{3}^{b_{3}}\cdots\xi_{m}^{b_{m}}) & =k\pi_{1}\end{align*}
The RHS is the long side. 
\end{itemize}

\pause{}
\begin{itemize}
\item Again, we use the fact we stated and observe that $\xi_{1}$divides
$\alpha_{1}\cdots\alpha_{s}$$\beta_{1}\cdots\beta_{t}$, and in particular,
$\xi_{1}$divides $(\pi_{1}^{a_{1}}-\xi_{1}^{b_{1}})$ or $(\pi_{1}^{a_{1}}\pi_{2}^{a_{2}}\cdots\pi_{k}^{a_{k}}-\xi_{2}^{b_{2}}\xi_{3}^{b_{3}}\cdots\xi_{m}^{b_{m}}).$
\item This means that $\xi_{1}$divides $\pi_{1}^{a_{1}}$ or $\xi_{2}^{b_{2}}\xi_{3}^{b_{3}}\cdots\xi_{m}^{b_{m}}$.
\end{itemize}

\lyxframeend{}\lyxframe{The Proof}
\begin{itemize}
\item If $\xi_{1}$ divides $\pi_{1}^{a_{1}}$ then $\pi_{1}^{a_{1}}=c\xi_{1}$.
The factorizations are of different lengths. As we have done before,
this leads to a contradiction since $\pi_{1}$ is not associated to
the other $\pi_{i}^{'}s$. 
\end{itemize}

\pause{}
\begin{itemize}
\item Otherwise, $c\xi_{1}=\xi_{2}^{b_{2}}\xi_{3}^{b_{3}}\cdots\xi_{m}^{b_{m}}$
\end{itemize}

\pause{}
\begin{itemize}
\item This means that $\xi_{1}$is associated with one of the $\xi_{i}^{'}s$
which is a contradiction.
\end{itemize}

\lyxframeend{}\lyxframe{The Proof}
\begin{itemize}
\item Therefore, we have no MF's of this form.
\end{itemize}

\pause{}
\begin{itemize}
\item The proof for the other two cases are similar. 
\end{itemize}

\lyxframeend{}

\appendix

\lyxframeend{}\section*{Appendix}


\lyxframeend{}\subsection*{For Further Reading}


\lyxframeend{}\lyxframe{[allowframebreaks]For Further Reading}

\beamertemplatebookbibitems
\begin{thebibliography}{2}
\bibitem{Coykendall,Smith1990}Coykendall, Jim and Smith, William. \newblock\emph{On Unique Factorization Domains}.\newblock
Journal of Algebra, 2011.\beamertemplatearticlebibitems



\end{thebibliography}

\lyxframeend{}
\end{document}
